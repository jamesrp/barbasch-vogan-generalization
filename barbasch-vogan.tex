\documentclass[12pt]{amsart}
\usepackage{amssymb,amsmath}
%\usepackage[left=1in,right=1in]{geometry}
\usepackage{amsthm}
\theoremstyle{plain}
\newtheorem{theorem}{Theorem}[section]
\newtheorem{lemma}[theorem]{Lemma}
\newtheorem{corollary}[theorem]{Corollary}
\newtheorem{proposition}[theorem]{Proposition}

\theoremstyle{definition}
\newtheorem{definition}[theorem]{Definition}
\newtheorem{example}[theorem]{Example}

\theoremstyle{remark}
\newtheorem{remark}[theorem]{Remark}
\newcommand{\Ind}{\textup{Ind}}
\newcommand{\Res}{\textup{Res}}
\newcommand{\wreath}{\textup{wr}}



\title{Barbasch-Vogan generalization}

\begin{document}
\maketitle
\section{Original Barbasch-Vogan proof}
Let $H_n \subset S_{2n}$ be the hyperoctahedral group. The aim of this paper is to reproduce the result of Barbasch and Vogan describing the representation $V_n := \Ind_{H_n}^{S_{2n}}(1)$.
\begin{lemma}
$S_{2n}/H_n \cong S_{2n-1}/(H_n \cap S_{2n-1}) = S_{2n-1}/H_{n-1}$ as sets of cosets with a $S_{2n-1}$ action. 
\end{lemma}
\begin{proof}
The second equality follows from $H_n \cap S_{2n-1} = H_{n-1}$. For the first, define a map $\phi: S_{2n} / H_n \to S_{2n-1}/H_{n-1}$ by $\phi(gH_n) = (gH_n) \cap S_{2n-1}$. When defining $\phi$, choosing the coset representative $g \in S_{2n-1}$ shows that $\phi$ is well-defined. It's straightforward to check that the $S_{2n-1}$ action commutes with $\phi$.
\end{proof}

\begin{lemma}
$\Res_{S_{2n-1}}^{S_{2n}}(V_n) = \Ind_{S_{2n-2}}^{S_{2n-1}}(V_{n-1})$.
\end{lemma}
\begin{proof}
By Lemma 1.1, $\Res_{S_{2n-1}}^{S_{2n}}(V_n) = S_{2n-1}/H_{n-1}$. But this is
\newline $ \Ind_{H_{n-1}}^{S_{2n-1}} 1 = \Ind_{S_{2n-2}}^{S_{2n-1}}(V_{n-1})$.
\end{proof}

\begin{theorem}
$V_n \cong \oplus S^\lambda$, summed over each $\lambda \vdash 2n$ with all parts even.
\end{theorem}
\begin{proof}
This is true for $n=1$. We use induction. Assume $V_{n-1}$ has the described decomposition. We use Lemma 1.2. By the branching rule, $\Ind^{S_{2n-1}}(V_{n-1})$ contains each $\mu \vdash 2n-1$ with exactly one odd part. Suppose $V_n$ contains $\lambda$ with at least three rows and at least two odd parts. Then the restriction of $\lambda$ contains a $\mu$ with three odd parts; thus these $\lambda$ do not occur. To rule out $\lambda$ with two odd rows, note that $\lambda=(2n)$ occurs in $V_n$ by Frobenius reciprocity. Therefore $(2n-1,1)$ can't occur, as it would contribute a second copy of $(2n-1)$. An induction on $i$ shows that $(2n-i,i)$ occurs if and only if $i$ is even.
\end{proof}

\section{Generalizing to other groups}
Our next goal is to treat the cases of $W_n = \Ind_{S_n \wreath S_m}^{S_{nm}}(1)$ and $Z_n = \Ind_{S_n \wreath C_m}^{S_{nm}}(1)$. In different ways, these each generalize the Barbasch-Vogan result. The strategy used above can be adapted to these cases.

\begin{lemma}
$\Res_{S_{nm-1}}^{S_{nm}}(W_n) = \Ind_{S_{(n-1)m} \times S_{m-1}}^{S_{nm-1}}(W_{n-1} \otimes 1)$.
$\Res_{S_{nm-1}}^{S_{nm}}(Z_n) = \Ind_{S_{(n-1)m} }^{S_{nm-1}}(Z_{n-1})$. 
The right-hand sides of these can be computed by applying Pieri's rule and the branching rule ($m-1$ times), respectively.
\end{lemma}
\begin{proof}
The proof is a straightforward generalization of Lemmas 1.1 and 1.2.
We intersect with $S_{nm-1}$. In the $Z_n$ case, this gives $S_{m-1} \wr Z_n$. 
In the $W_n$ case, it gives $(S_{m-1} \wr S_n) \times S_{n-1}$.
Therefore, in the $Z_n$ case, we use the inclusion $S_{n(m-1)} \subseteq S_{nm-1}$, while in the $W_n$ case, we use the inclusion $S_{n(m-1)} \times S_{n-1} 
\subseteq S_{nm-1}$. These corresponding to applying Pieri's rule repeatedly
with $k=1$, and applying it once with $k=n-1$, respectively.
\end{proof}
\end{document}
