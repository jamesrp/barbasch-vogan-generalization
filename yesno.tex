\documentclass{article}
\usepackage{amsthm}
\theoremstyle{plain}
\newtheorem{theorem}{Theorem}[section]
\newtheorem{lemma}[theorem]{Lemma}
\newtheorem{corollary}[theorem]{Corollary}
\newtheorem{proposition}[theorem]{Proposition}

\theoremstyle{definition}
\newtheorem{definition}[theorem]{Definition}
\newtheorem{example}[theorem]{Example}

\newcommand{\mult}{\textup{mult}}

\begin{document}
We attempt to classify the partitions appearing in $m=3$,$n = 1, 2, \ldots$.
\section{Constructing yes and no lists}
We denote partitions by their second through last elements.
Thus, a partition here is a pattern that exists for any $n$.

First, note that the empty list 0 is present exactly once at each level $n$.

We also denote the intermediate level, branch-down($n$), by $n^-$. 
Thus, we can write notation like $\mult(2,n)$ and $\mult(2,n^-)$ for the multiplicities of 2 in $n$ and in $n^-$.

\begin{lemma}
For all $n$, $\mult(0,n)=1$.
\end{lemma}
\begin{proof}
We use induction on $n$. For $n=1$ this is true. Suppose it holds for $n-1$.
Then $\mult(0,n^-)=1$.
We know from the definition of our module that $\mult(0,n) \ge 1$ for each $n$ (as there exists an invariant element).
This forces $\mult(0,n)=1$; otherwise, $\mult(0,n^-) \ge 2$.
\end{proof}

\begin{lemma}
For all $j \ge 1$ and all $n$, $1^j$ does not appear.
\end{lemma}
\begin{proof}
Induction on $j$. 
For $j = 1$ we use induction on $n$.
For the base case, note that level $1$ does not contain the partition $1$.
Suppose $p = 1$ appears at level $n$.
Then 1 and 0 appear at level $n^-$.
Assume 1 does not appear at level $n-1$. Then 0 appears twice at level $n-1$, contradiction.

The argument for higher $j$ is similar; if $p=1^j$ appears at level $n$, then $1^j$ and $1^{j-1}$ appear at level $n^-$. 
But $1^{j-1}$ implies one of $1^{j-1}$ or $1^{j-2}$ appear at level $n-1$, contradiction.

\end{proof}

\begin{lemma}
For all $n$ and all $i\ge 0, j \ge 1$, $2^i1^j$ does not appear.
\end{lemma}
\begin{proof}
For $i=0$ this is Lemma 1.1.
Then use induction on $i$ (and within, induction on $n$).
Note that $2^i1^j$ does not appear at $n=1$.
Suppose it appears at level $n$.
Then $2^{i-1}1^{j+1}$ appears at level $n^-$.
The possible elements at level $n-1$ that explain this are:
$$
2^{i-1}1^{j+1},
2^{i-2}1^{j+2},
2^{i-1}1^{j},
2^{i-2}1^{j+1}.
$$
None of these appear at level $n-1$ by induction on $i$.
\end{proof}

\begin{lemma}
The multiplicity of 2 is 1 for $n \ge 2$.
\end{lemma}
\begin{proof}
The only way for 1 to be in $n^-$ is by inducing from $0 \in n-1$. Thus, $\mult(1,n^-)=1$. Thus, exactly one of $1,2,11 \in n$. Since 1 and 11 don't appear, we have $\mult(2)=1$.
\end{proof}

\begin{lemma}
311 does not appear for any $n$.
\end{lemma}
\begin{proof}
If 311 appears at level $n$, then 211 appears at level $n^-$; we have seen that this leads to a contradiction.
\end{proof}

\begin{lemma}
For all $n \ge 3$, $\mult(3,n) = 1$.
\end{lemma}
\begin{proof}
First, note that $\mult(2,n^-)=2$ for $n \ge 3$, since each copy arises from 0 or 2 in $n-1$, and we know each of those appears once.
Now, each $2 \in n^-$ implies one of $2,3 \in n$. Since $\mult(2,n)=1$, we must have $\mult(3,n)=1$ as well.
\end{proof}

\begin{lemma}

\end{lemma}
\begin{proof}

\end{proof}


\end{document}
